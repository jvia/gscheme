\documentclass{article}
\usepackage{bussproofs}
\usepackage{mathtools}
\usepackage{fullpage}
\usepackage{fancyhdr}
\fancyhead[EOR]{Jeremiah Via --- Homework 02}
\fancyhead[EOL]{}

%% Macros
\newcommand{\STATE}[4]{\langle #1, #2, #3, #4 \rangle}%
\newcommand{\VAR}[1]{\text{VAR}(#1)}%
\newcommand{\SET}[2]{\text{SET}(#1, #2)}%
\newcommand{\IF}[3]{\text{IF}(#1, #2, #3)}%
\newcommand{\DOM}[1]{\text{dom } #1}%
\newcommand{\LITERAL}[1]{\text{LITERAL}(#1)}%
\newcommand{\FOR}[4]{\text{FOR}(#1,#2,#3,#4)}%
\newcommand{\EVAL}[0]{\Downarrow}%
\newcommand{\APPLY}[2]{\text{APPLY}(#1,#2)}%
\newcommand{\LIST}[1]{\text{LIST}(#1)}%
\newcommand{\TYPEENV}[1]{\Gamma_\xi,\Gamma_\phi,\Gamma_\rho \vdash\;}%

\author{Jeremiah Via}
\title{COMP105 --- Homework 02}

\begin{document}
\maketitle
\thispagestyle{empty}
\pagestyle{fancy}



%%%%%%%%%%%%%%%%%%%%%%%%%%%%%%%%%%%%%%%%%%%%%%%%%%%%%%%%%%%%%%%%%%%%%% 
% Formation
%%%%%%%%%%%%%%%%%%%%%%%%%%%%%%%%%%%%%%%%%%%%%%%%%%%%%%%%%%%%%%%%%%%%%% 
\section{Formation}

\begin{prooftree}
  \AxiomC{$\tau \text{ is a type}$}
  \RightLabel{\textsc{(ListFormation)}}
  \UnaryInfC{$\LIST{\tau} \text{ is a type}$}
\end{prooftree}

%%%%%%%%%%%%%%%%%%%%%%%%%%%%%%%%%%%%%%%%%%%%%%%%%%%%%%%%%%%%%%%%%%%%%% 
% Introduction
%%%%%%%%%%%%%%%%%%%%%%%%%%%%%%%%%%%%%%%%%%%%%%%%%%%%%%%%%%%%%%%%%%%%%% 
\section{Introduction}

%%%%%%%%%%%%%%%%%%%%%%%%%%%%%%%%%%%%%%%%%%%%%%%%%%%%%%%%%%%%%%%%%%%%%% 
% Elimination
%%%%%%%%%%%%%%%%%%%%%%%%%%%%%%%%%%%%%%%%%%%%%%%%%%%%%%%%%%%%%%%%%%%%%% 
\section{Elimination}

\begin{prooftree}
  \AxiomC{$\TYPEENV e : \LIST{\tau}$}
  \UnaryInfC{$\TYPEENV \text{CAR}(e) : \tau$}
\end{prooftree}

\section*{Question 10}
\begin{prooftree}
  %% PREMISE LEFT SIDE
  \AxiomC{$x \notin \text{dom }p_0$}
  \AxiomC{$x \in \text{dom }\xi_0$}
  \AxiomC{}
  \UnaryInfC{$\langle\text{LITERAL}(3), \xi_0, \phi, \rho_0 \rangle$}
  \TrinaryInfC{$\langle\text{SET } x \; 3, \xi_0, \phi, \rho_0 \rangle \Downarrow \langle 3, \xi_1\{x \rightarrow 3\}, \phi, \rho_1 \rangle$}
  %% PREMISE RIGHT SIDE
  \AxiomC{$x \notin \text{dom } \rho_1$}
  \AxiomC{$x \in \text{dom } \xi_1$}
  \BinaryInfC{$\langle \text{VAR}(x), \xi_1, \phi, \rho_1\rangle \Downarrow \langle 3, \xi_1, \phi, \rho_0 \rangle$}
  %% CONCLUSION
  \BinaryInfC{$\langle\text{BEGIN}(\text{SET } x\; 3, \text{VAR}(x)), \xi_0, \phi, \rho_0 \rangle   \Downarrow \langle 3, \xi_1, \phi, \rho_1 \rangle$}
\end{prooftree}

%%%%%%%%%%%%%%%%%%%%%%%%%%%%%%%%%%%%%%%%%%%%%%%%%%%%%%%%%%%%%%%%%%%%%% 
% Question 11 
%%%%%%%%%%%%%%%%%%%%%%%%%%%%%%%%%%%%%%%%%%%%%%%%%%%%%%%%%%%%%%%%%%%%%% 
\newpage
\section*{Question 11}
If
\begin{prooftree}
  \AxiomC{$x \notin \DOM{\rho}$}
  \AxiomC{$x \in \xi$}
  \BinaryInfC{$\STATE{\VAR{x}}{\xi}{\phi}{\rho} \Downarrow \STATE{v_1}{\xi}{\phi}{\rho}$}

  \AxiomC{$v_1 \neq 0$}

  \AxiomC{$x \notin \DOM{\rho}$}
  \AxiomC{$x \in \DOM{\rho}$}
  \BinaryInfC{$\STATE{\VAR{x}}{\xi}{\phi}{\rho} \Downarrow \STATE{v_1}{\xi}{\phi}{\rho}$}

  \TrinaryInfC{$\STATE{\IF{\VAR{x}}{\VAR{x}}{\LITERAL{0}}}{\xi}{\phi}{\rho} \Downarrow \STATE{v_1}{\xi}{\phi}{\rho}$}
\end{prooftree}
or
\begin{prooftree}
  \AxiomC{$x \notin \DOM{\rho}$}
  \AxiomC{$x \in \xi$}
  \BinaryInfC{$\STATE{\VAR{x}}{\xi}{\phi}{\rho} \Downarrow \STATE{v_1}{\xi}{\phi}{\rho}$}

  \AxiomC{$v_1 = 0$}

  \AxiomC{}
  \UnaryInfC{$\STATE{\LITERAL{0}}{\xi}{\phi}{\rho} \Downarrow \STATE{0}{\xi}{\phi}{\rho}$}

  \TrinaryInfC{$\STATE{\IF{\VAR{x}}{\VAR{x}}{\LITERAL{0}}}{\xi}{\phi}{\rho} \Downarrow \STATE{v_1}{\xi}{\phi}{\rho}$}
\end{prooftree}
and
\begin{prooftree}
  \AxiomC{$x \notin \DOM{\rho}$}
  \AxiomC{$x \in \DOM{\phi}$}
  \BinaryInfC{$\STATE{\VAR{x}}{\xi}{\phi}{\rho} \Downarrow \STATE{v_2}{\xi'}{\phi}{\rho'}$}
\end{prooftree}
then
$v_1=v_2$
%%%%%%%%%%%%%%%%%%%%%%%%%%%%%%%%%%%%%%%%%%%%%%%%%%%%%%%%%%%%%%%%%%%%%% 
% Question 13
%%%%%%%%%%%%%%%%%%%%%%%%%%%%%%%%%%%%%%%%%%%%%%%%%%%%%%%%%%%%%%%%%%%%%% 
\newpage
\section*{Question 13}
\subsection*{a)}

\begin{prooftree}
  \AxiomC{$x \notin \DOM{\rho}$}
  \AxiomC{$x \notin \DOM{\xi}$}
  \RightLabel{\textsc{(AwkVar)}}
  \BinaryInfC{$\STATE{\VAR{x}}{\xi}{\phi}{\rho} \Downarrow \STATE{0}{\xi'\{x \mapsto 0\}}{\phi}{\rho}$}
\end{prooftree}
\vspace{1cm}
\begin{prooftree}
  \AxiomC{$x \notin \DOM{\rho}$}
  \AxiomC{$x \notin \DOM{\xi}$}
  \AxiomC{$\STATE{e}{\xi}{\phi}{\rho} \Downarrow \STATE{v}{\xi'}{\phi}{\rho'}$}
  \RightLabel{\textsc{(AwkSet)}}
  \TrinaryInfC{$\STATE{\SET{x}{e}}{\xi}{\phi}{\rho} \Downarrow \STATE{v}{\xi'\{x \mapsto v\}}{\phi}{\rho'}$}
\end{prooftree}

\subsection*{b)}

\begin{prooftree}
  \AxiomC{$x \notin \DOM{\rho}$}
  \AxiomC{$x \notin \DOM{\xi}$}
  \RightLabel{\textsc{(IconVar)}}
  \BinaryInfC{$\STATE{\VAR{x}}{\xi}{\phi}{\rho} \Downarrow \STATE{0}{\xi}{\phi}{\rho'\{x \mapsto 0\}}$}
\end{prooftree}
\vspace{1cm}
\begin{prooftree}
  \AxiomC{$x \notin \DOM{\rho}$}
  \AxiomC{$x \notin \DOM{\xi}$}
  \AxiomC{$\STATE{e}{\xi}{\phi}{\rho} \Downarrow \STATE{v}{\xi'}{\phi}{\rho'}$}
  \RightLabel{\textsc{(IconSet)}}
  \TrinaryInfC{$\STATE{\SET{x}{e}}{\xi}{\phi}{\rho} \Downarrow \STATE{v}{\xi'}{\phi}{\rho'\{x \mapsto v\}}$}
\end{prooftree}

\subsection*{c)}
I prefer Icon because implicitly created variables have a shorter
scope. In any sufficiently sized program, this behavior will be easier
to reason about.

%%%%%%%%%%%%%%%%%%%%%%%%%%%%%%%%%%%%%%%%%%%%%%%%%%%%%%%%%%%%%%%%%%%%%% 
% Question 16
%%%%%%%%%%%%%%%%%%%%%%%%%%%%%%%%%%%%%%%%%%%%%%%%%%%%%%%%%%%%%%%%%%%%%% 
\newpage
\section*{Question 16}
Yes, it is possible to modify the environment $\rho$ without a
\textsc{Set} node. A trivial example that demonstrates this behavior
is applying the identity function $I$ to the literal 1. 

\begin{prooftree}
  \AxiomC{$\phi(I) = USER(\langle x \rangle, e)$}

  \AxiomC{}
  \UnaryInfC{$\STATE{\LITERAL{1}}{\xi}{\phi}{\rho_0} \EVAL \STATE{1}{\xi}{\phi}{\rho_0}$}

  \AxiomC{$\STATE{e}{\xi}{\phi}{\rho_0\{x \mapsto 1\}} \EVAL \STATE{1}{\xi}{\phi}{\rho_1}$}
  \TrinaryInfC{$\STATE{\APPLY{I}{\LITERAL{1}}}{\xi}{\phi}{\rho_0} \EVAL \STATE{1}{\xi}{\phi}{\rho_1}$}
\end{prooftree}

In this trivial case, we can see that $\rho$ is modified and thus it
is possible to modify an environment without explicitly using
\textsc{Set}.

%%%%%%%%%%%%%%%%%%%%%%%%%%%%%%%%%%%%%%%%%%%%%%%%%%%%%%%%%%%%%%%%%%%%%% 
% Question 17
%%%%%%%%%%%%%%%%%%%%%%%%%%%%%%%%%%%%%%%%%%%%%%%%%%%%%%%%%%%%%%%%%%%%%% 
\newpage
\section*{Question 17}

\begin{prooftree}
  \AxiomC{$\STATE{e_2}{\xi}{\phi}{\rho} \Downarrow \STATE{v_1}{\xi'}{\phi}{\rho'}$}
  \AxiomC{$v_1=0$}
  \RightLabel{\textsc{(ForEnd)}}
  \BinaryInfC{$\STATE{\FOR{e_1}{e_2}{e_3}{e_4}}{\xi}{\phi}{\rho} \Downarrow \STATE{0}{\xi'}{\phi}{\rho'}$}  
\end{prooftree}

\vspace{1cm}

\begin{prooftree}
  %% e1 & e2 & v
  \AxiomC{$\STATE{e_1}{\xi_1}{\phi}{\rho_1} \EVAL \STATE{v_1}{\xi_2}{\phi}{\rho_2}$}
  \noLine
  \UnaryInfC{$\STATE{e_2}{\xi_2}{\phi}{\rho_2} \EVAL \STATE{v_2}{\xi_3}{\phi}{\rho_3}$}
  \noLine
  \UnaryInfC{$v_2 \neq 0$}
  \noLine
  %% e3 & e4
  \UnaryInfC{$\STATE{e_4}{\xi_3}{\phi}{\rho_3} \EVAL \STATE{v_3}{\xi_4}{\phi}{\rho_4}$}
  \noLine
  \UnaryInfC{$\STATE{e_3}{\xi_4}{\phi}{\rho_4} \EVAL \STATE{v_4}{\xi_5}{\phi}{\rho_5}$}


  %% recursion
  \AxiomC{$\STATE{\FOR{e_1}{e_2}{e_3}{e_4}}{\xi_5}{\phi}{\rho_5} \EVAL \STATE{v_5}{\xi_6}{\phi}{\rho_6}$}
  \RightLabel{\textsc{(ForIterate)}}
  \BinaryInfC{$\STATE{\FOR{e_1}{e_2}{e_3}{e_4}}{\xi_1}{\phi}{\rho_1} \Downarrow \STATE{v_5}{\xi_6}{\phi}{\rho_6}$}
\end{prooftree}

\end{document}
